\documentclass[a4paper, 12pt]{extarticle}

%%% Includes 
\usepackage[utf8]{inputenc} % UTF-8 encode 
\usepackage[english, russian]{babel}
\usepackage{geometry} % adjust page layout 
\usepackage{graphicx} 
\usepackage{hyperref} 
\usepackage{amsmath} % math formulas 
\usepackage{amsthm} % theorems, definitions, etc.
\usepackage{setspace} % for set line spacing 
\usepackage{indentfirst} % indent on a first line after the paragraph 
% \usepackage{pgfplots} % for plots 
\usepackage{listings} % for code listings 
\usepackage{xcolor} % colors (used for listings)
\usepackage{sourcecodepro} % for another monospaced font 
\usepackage{cmap} % for correct search in pdf
\usepackage{placeins} % for \FloatBarrier
\usepackage{enumitem} % for custom lists
\usepackage{float}   
\usepackage{subcaption} % Для создания подрисунков (subfigure)
\usepackage{listings}

\usepackage{array}
\usepackage{booktabs} % Для профессионального оформления таблиц
\usepackage{siunitx} % Для правильного отображения чисел и единиц измерения
%%% Debug
% \usepackage{showframe} % frame borders for demonstration 
\geometry{left=2cm, right=2cm, bottom=2cm, top=2cm}

% Настраиваем супер красивые листинги с попомщью пакетов listings и xcolor
\definecolor{codegreen}{rgb}{0,0.6,0}
\definecolor{codegray}{rgb}{0.5,0.5,0.5}
\definecolor{codepurple}{rgb}{0.58,0,0.82}
\definecolor{backcolour}{rgb}{0.95,0.95,0.92}

\lstdefinestyle{codestyle}{
    backgroundcolor=\color{backcolour},
    commentstyle=\color{codegreen},
    keywordstyle=\color{magenta},
    numberstyle=\tiny\color{codegray},
    stringstyle=\color{codepurple},
    basicstyle=\ttfamily\footnotesize,
    breakatwhitespace=false,
    breaklines=true,
    captionpos=b,
    keepspaces=true,
    numbers=left,
    numbersep=5pt,
    showspaces=false,
    showstringspaces=false,
    showtabs=false,
    tabsize=2
}

\lstset{style=codestyle}
\lstset{extendedchars=\true}

\makeatletter
\renewcommand{\@seccntformat}[1]{}
\makeatother

\usepackage{tikz}
\usetikzlibrary{matrix}

%%% Custom commands
% commands for unnumbered sections
\newcommand{\usection}[1]{\section*{#1} \addcontentsline{toc}{section}{\protect\numberline{}#1}}
\newcommand{\usubsection}[1]{\subsection*{#1} \addcontentsline{toc}{subsection}{\protect\numberline{}#1}}
\newcommand{\usubsubsection}[1]{\subsubsection*{#1} \addcontentsline{toc}{subsubsection}{\protect\numberline{}#1}}

% math commands
\theoremstyle{definition}
\newtheorem{definition}{Определение}

\theoremstyle{plain}
\newtheorem{theorem}{Теорема}
\newtheorem*{thproof}{Доказательство}
\newtheorem*{example}{Пример}

\theoremstyle{remark}
\newtheorem*{notabene}{Замечание}
\newtheorem*{solution}{Решение}


% Redefinition of section and subsection numbering style (with dot at the end)
\def\thesection{\arabic{section}.}
\def\thesubsection{\arabic{section}.\arabic{subsection}.}
\def\thesubsubsection{\arabic{section}.\arabic{subsection}.\arabic{subsubsection}.}

% Theorems, definitions, etc.



%%% Settings for links 
\hypersetup{
    colorlinks,
    citecolor=black,
    filecolor=black,
    linkcolor=blue,
    urlcolor=blue
}


%%% Layout
\geometry{
	left=17mm, % left margin
	top=17mm, % top margin
	right=17mm, % right margin
	bottom=20mm, % bottom margin
	marginparsep=0mm, % space between text and margin notes
	marginparwidth=0mm, % width of margin notes
	headheight=8mm, % height of the header
	headsep=5mm, % space between header and text
}


\linespread{1.5} % line spacing
\setlength{\parskip}{\baselineskip}  % Add space between paragraphs

\setlist{itemsep=1pt,topsep=1pt,parsep=1pt} % custom list settings


% overfull hbox settings
\tolerance 5000 % default 200, max 10000, 
\hbadness 3000 % default 1000, max 10000, warning threshold for underfull hbox
\emergencystretch 0pt  % default 0pt, how much the lines can stretch for the sake of good line breaks
\hfuzz 0.4pt % ignore overfull box less than 
\widowpenalty=10000 % no lines at the start of the page
\vfuzz \hfuzz % don't care about underfull vbox if overfull is acceptable
\raggedbottom % if the page is not filled, align the content to the bottom


%%% Redefinition of table of contents command to get centered heading
\makeatletter
\renewcommand\tableofcontents{ 
  \begin{singlespace}
    \null\hfill\textbf{\Large\contentsname}\hfill\null\par
    \@mkboth{\MakeUppercase\contentsname}{\MakeUppercase\contentsname}%
    \@starttoc{toc}
  \end{singlespace}
}
\makeatother


%%% Listings settings
\definecolor{codegreen}{rgb}{0, 0.6, 0}
\definecolor{codegray}{rgb}{0.5, 0.5, 0.5}
\definecolor{codepurple}{rgb}{0.58, 0, 0.82}
\definecolor{backcolour_gray}{rgb}{0.98, 0.98, 0.98}

\lstdefinestyle{python_white}{
  language=Python,
  backgroundcolor=\color{backcolour_gray},   
  commentstyle=\color{codegreen},
  keywordstyle=\color{blue},
  numberstyle=\tiny\color{codegray},
  stringstyle=\color{codepurple},
  basicstyle=\ttfamily\small\singlespacing,
  breakatwhitespace=true,         
  breaklines=true,                 
  captionpos=b, % t/b                  
  keepspaces=true,                 
  numbers=none, % none/left/rigth                    
  numbersep=5pt,                  
  showspaces=false,                
  showstringspaces=false,
  showtabs=false,                  
  tabsize=2,
  frame=single, % none/leftline/topline/bottomline/lines/single/shadowbox
  rulecolor=\color{gray}, % frame color 
}

\lstset{style=python_white} % set default listings style



%%% For title page
\def\name{Отчет по лабораторной работе 0} 
\def\matter{Машинное обучение}
\def\madeby{Ворожцов Константин Александрович, R3342}
\def\teacher{Прокопов Егор Максимович}

\begin{document}

\include{unformal_title} % Title page

\addtocounter{page}{1} % Inc counter to start from 2 

\section{Задание 1}
\subsection*{Цель: поразмышлять о своих целях и задачах на текущий семестр}

\begin{enumerate}[label=\arabic*.]
    \item С помощью веб-интерфейса GitHub создайте файл \texttt{goals.md} в своем репозитории.
    
    \item Используя язык разметки markdown, создайте заголовки ``Мой опыт'' и ``Мои цели''.
    
    \item Напишите один абзац (1--5 предложений) о своем опыте по разработке программного обеспечения. Добавьте работающую ссылку в написанный текст (например, на ваш профиль в GitHub).
    
    \item Создайте нумерованный список из 2 или 3 целей, которых планируете достичь в рамках данного курса.
    
    \item Создайте ненумерованный список из 3--9 задач, над которыми собираетесь работать в рамках данного курса для достижения поставленных целей.
    
    \item Создайте папку \texttt{resources} в корне репозитория. Загрузите в эту папку изображение (\texttt{.png/jpg/jpeg}) и добавьте его в ваш \texttt{.md} файл.
    
    \item Не забудьте закоммитить файлы в свой репозиторий на GitHub.
    
    \item Проверьте, что созданный \texttt{.md} файл отформатирован и отображается при предпросмотре именно так, как планировалось.
\end{enumerate}

\subsection{Результаты}

Содержимое файлов goals.md, info.md, .gitignore представлено в приложении.

Объяснение результата попытки закоммитить файл data/dataset.csv: Так как мы добавили путь файла в .gitignore, то все что находится по этому пути будет игнорироваться git, и не добавляться в коммит.

\newpage
\begin{lstlisting}[language=markdown, caption = {Содержимое файла goals.md }]
# Мой опыт

 У меня есть опыт написания Python/Matlab скриптов в учебных целях, а именно для решения задач численного моделирования, математической оптимизации и Фурье анализа. Есть опыт создания модели [трехфазного инвертора](https://sibcontact.com/eshop/preobrazovateli-napryazheniya/vhodnoe-napryazhenie-12v/is3-12-600-invertor-dc-ac-12-v-600-vt/?utm_source=yandex&utm_medium=cpc&utm_campaign=kry_dpo_obshaja_poisk&utm_content=ch_yandex_direct%7Ccid_114689624%7Cgid_5498556019%7Cad_1855295429420572829%7Cph_53243354126%7Ccrt_0%7Cpst_premium%7Cps_2%7Csrct_search%7Csrc_none%7Cdevt_desktop%7Cret_53243354126%7Cgeo_2%7Ccf_0%7Cint_%7Ctgt_53243354126%7Cadd_no%7Cdop_&utm_term=---autotargeting&ybaip=1&yclid=14699200476502294527) с использованием Simulink.

# Мои цели

1. Обрести понимание того, как происходит машинное обучение.
2. Разобраться в различных существующих принципах машинного обучения.
3. Понять как можно использовать методы машинного обучения в свои будущих проектах.

#### Задачи, над которыми надо работать в рамках данного курса для достижения поставленных целей.

- Вдумчивое самостоятельное выполнение лабораторных работ.
- Разбор всех материалов лекций.
- Ведение кратких конспектов по каждому из методов машинного обучения.

![screen](https://github.com/Kostya545215/Machine-Learning-ITMO/blob/main/resources/schema11.jpg)
\end{lstlisting}

\begin{lstlisting}[language = md, caption ={Содержимое файла info.md}]
 Ворожцов Константин Александрович 

Microsoft Windows [Version 10.0.19045.6216]

15.09.2025
    
\end{lstlisting}

\begin{lstlisting}[language = md, caption ={Содержимое файла .gitignore}]
data/
\end{lstlisting}
\end{document}